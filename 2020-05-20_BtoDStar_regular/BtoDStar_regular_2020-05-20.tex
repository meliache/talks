% Created 2020-05-20 Mi 16:40
% Intended LaTeX compiler: xelatex
\documentclass[aspectratio=169, 16pt]{beamer}
\usepackage{graphicx}
\usepackage{grffile}
\usepackage{longtable}
\usepackage{wrapfig}
\usepackage{rotating}
\usepackage[normalem]{ulem}
\usepackage{amsmath}
\usepackage{textcomp}
\usepackage{amssymb}
\usepackage{capt-of}
\usepackage{hyperref}
\usepackage{xcolor}
\usepackage{physics}
\usepackage{siunitx}
\usepackage{booktabs}
\usepackage[babel]{microtype}
\newcommand{\PDmstar}{\HepParticle{D}{}{\left(*\right)}}
\author{Michael Eliachevitch}
\date{\today}
\title{My status and ideas for collaboration}
\subtitle{Regular meeting to discuss hadronically tagged \PB \rightarrow \PDmstar \ell \nu_\ell}
\usepackage{templates/metropolisbonn}
\usepackage{hepnames, hepparticles}
\usepackage[mode=build]{standalone}
\institute{Physikalisches Institut --- Rheinische Friedrich-Wilhelms-Universität Bonn}
\hypersetup{colorlinks, urlcolor=bonnblue}
\lstset{keywordstyle=\bfseries\color{bonnblue}, commentstyle=\itshape\color{bonnunigrau}, identifierstyle=\color{bonntextgrau}, stringstyle=\color{bonnyellow}}
\hypersetup{
 pdfauthor={Michael Eliachevitch},
 pdftitle={My status and ideas for collaboration},
 pdfkeywords={},
 pdfsubject={},
 pdfcreator={Emacs 28.0.50 (Org mode 9.3.6)}, 
 pdflang={English}}
\begin{document}

\maketitle
\begin{frame}[label={sec:orge64fa0b},fragile]{My analysis project status}
 \begin{itemize}
\item working on another package for hadronically tagged \(\PB \rightarrow \PDmstar \ell
  \nu_\ell\) decays
\item started as a fork of Felix Metzner's \texttt{b2bii}-based \texttt{RDStar} framwork for Belle
\item recently decoupled my framework, made it a separate python
package, repository now public on \href{https://stash.desy.de/projects/B2A/repos/meliache\_b2\_rdstar}{B2A: \texttt{meliache\_b2\_rdstar}}
\item where useful re-use existing functionality from \texttt{RDStar} and the \texttt{AnalysisTools}
packages (via python imports)
\item adapt selection to Belle II data
\item use b2luigi with my \href{https://meliache.github.io/b2luigi/docs/\_build/html/usage/batch.html\#gbasf2-wrapper-for-lcg-jobs}{gbasf2 extension} for grid job submission
\item status: implemented all \PD and \PDstar decay channels and preparing to produce my own
ntuples using MC13a FEI-skims
\end{itemize}
\end{frame}

\begin{frame}[label={sec:org64e2821}]{Ideas for collaboration}
Offer help / assistance e.g. in: 
\begin{itemize}
\item \(M_{miss}^2\) resolution
\item PID uncertainties
\item (template) fits
\item cross-checks: reproduce ntuples with my package and validate signal yield\\
(requires cut-flows for interpretation)
\end{itemize}
\end{frame}

\begin{frame}[label={sec:org21db1cc}]{Backup}
\appendix
\end{frame}
\begin{frame}[label={sec:org8444811},fragile]{Example b2luigi gbasf2 task}
 \lstset{language=Python,label= ,caption= ,captionpos=b,numbers=none,basicstyle=\scriptsize\ttfamily, xleftmargin=-5pt}
\begin{lstlisting}
import b2luigi
from b2luigi.basf2_helper.tasks import Basf2PathTask
from b2_rdstar.online_analysis import reconstruction

class OnlineReconstructionTask(Basf2PathTask):
    batch = "gbasf2"
    gbasf2_project_name_prefix = "OnlineReco"
    gbasf2_input_dataset = \
        "/belle/MC/release-04-01-04/DB00000774/SkimM13ax1/prod00011778/e1003/4S/r00000/mixed/11180100/udst/sub00"

    def create_path(self):
        return reconstruction.create_online_reconstruction_path()

    def output(self):
        yield self.add_to_output("D+:Kpipi_ntuple.root")
        yield self.add_to_output("B0:sig_ntuple.root")

if __name__ == '__main__':
    b2luigi.process(OnlineReconstructionTask())
\end{lstlisting}
\end{frame}
\end{document}