% Created 2020-05-20 Mi 00:44
% Intended LaTeX compiler: xelatex
\documentclass[aspectratio=169, 16pt]{beamer}
\usepackage{graphicx}
\usepackage{grffile}
\usepackage{longtable}
\usepackage{wrapfig}
\usepackage{rotating}
\usepackage[normalem]{ulem}
\usepackage{amsmath}
\usepackage{textcomp}
\usepackage{amssymb}
\usepackage{capt-of}
\usepackage{hyperref}
\usepackage{xcolor}
\usepackage{physics}
\usepackage{siunitx}
\usepackage{booktabs}
\usepackage[babel]{microtype}
\author{Michael Eliachevitch}
\date{\today}
\title{Weekly report}
\subtitle{BBAM! Meeting}
\usepackage{templates/metropolisbonn}
\usepackage{hepnames, hepparticles}
\usepackage[mode=build]{standalone}
\institute{Physikalisches Institut --- Rheinische Friedrich-Wilhelms-Universität Bonn}
\hypersetup{colorlinks, urlcolor=bonnblue}
\lstset{keywordstyle=\bfseries\color{bonnblue}, commentstyle=\itshape\color{bonnunigrau}, identifierstyle=\color{bonntextgrau}, stringstyle=\color{bonnyellow}}
\hypersetup{
 pdfauthor={Michael Eliachevitch},
 pdftitle={Weekly report},
 pdfkeywords={},
 pdfsubject={},
 pdfcreator={Emacs 28.0.50 (Org mode 9.3.6)}, 
 pdflang={English}}
\begin{document}

\maketitle
\begin{frame}[label={sec:org19ae89f}]{Using path serialization to send jobs to grid}
\centering
\includestandalone[width=0.8\textwidth]{figures/pickled_paths_to_gbasf2/pickled_paths_to_gbasf2}
\end{frame}
\begin{frame}[label={sec:org175e2ef},fragile]{gbasf2 b2luigi task}
 \begin{itemize}
\item base class, user can inheret and implement \texttt{create\_path()} fucntion
\item creates pickle and wrapper steering file
\item sends grid jobs via \texttt{gbasf2} and downloads them if all are \texttt{Done}
\item stores \texttt{gbasf2} environment in dictionary and uses it only in subprocess calls
to gbasf2 commands
\end{itemize}

\appendix
\end{frame}

\begin{frame}[label={sec:org00950e7},fragile]{Usage}
 \lstset{language=Python,label= ,caption= ,captionpos=b,numbers=none,basicstyle=\tiny\ttfamily, xleftmargin=-5pt}
\begin{lstlisting}
import b2luigi
from rdstar.online_analysis import reconstruction_belle2
from rdstar.online_analysis.gbasf2_tasks import Gbasf2Task


class MyGbasf2Task(Gbasf2Task):
    def create_path(self):
        return reconstruction_belle2.create_BtoDlnu_reconstruction_path()


class MasterTask(b2luigi.WrapperTask):
    def requires(self):
        yield MyGbasf2Task(
            max_event=100,
            project_name="luigi_testproject4",
            input_dataset="/belle/MC/release-04-00-03/DB00000757/MC13a/prod00009434/s00/e1003/4S/r00000/mixed/mdst/sub00/mdst_000255_prod00009434_task10020000255.root",
        )

if __name__ == '__main__':
    b2luigi.process(MasterTask())
\end{lstlisting}
\end{frame}
\begin{frame}[label={sec:orga73c833},fragile]{How environment is set up}
 \lstset{language=Python,label= ,caption= ,captionpos=b,numbers=none,basicstyle=\tiny\ttfamily, xleftmargin=-5pt}
\begin{lstlisting}
#: cached gbasf2 enviromnent, initiallized and accessed via ``self.env``
_cached_env = None

@property
def env(self) -> dict:
    """
    Return dictionary with gbasf2 environment from ``self.env_script``.

    Runs the script only once and then caches the result in ``self._cached_env``
    to prevent unnecessary password prompts.
    """
    if self._cached_env is None:
        print(f"Setting up environment by sourcing {self.env_script}")
        command = shlex.split(f"env -i bash -c 'source {self.env_script} > /dev/null && env'")
        output = subprocess.check_output(command, encoding="utf-8")
        self._cached_env = dict(line.split("=", 1) for line in output.splitlines())
    return self._cached_env
\end{lstlisting}
\end{frame}
\end{document}