% Created 2020-02-10 Mo 11:01
% Intended LaTeX compiler: xelatex
\documentclass[aspectratio=169, 16pt]{beamer}
\usepackage{graphicx}
\usepackage{grffile}
\usepackage{longtable}
\usepackage{wrapfig}
\usepackage{rotating}
\usepackage[normalem]{ulem}
\usepackage{amsmath}
\usepackage{textcomp}
\usepackage{amssymb}
\usepackage{capt-of}
\usepackage{hyperref}
\usepackage{xcolor}
\usepackage{physics}
\usepackage{siunitx}
\usepackage{booktabs}
\usepackage[babel]{microtype}
\author{Michael Eliachevitch}
\date{\today}
\title{Weekly report}
\subtitle{BAMM! Meeting}
\usepackage{templates/metropolisbonn}
\usepackage{hepnames, hepparticles}
\newcommand{\PDmstar}{\HepParticle{D}{}{\left(*\right)}}}
\newcommand{\rdstar}{R\left(\PDmstar\right)}
\institute{Physikalisches Institut --- Rheinische Friedrich-Wilhelms-Universität Bonn}
\hypersetup{colorlinks, urlcolor=bonnblue}
\lstset{keywordstyle=\bfseries\color{bonnblue}, commentstyle=\itshape\color{bonnunigrau}, identifierstyle=\color{bonntextgrau}, stringstyle=\color{bonnyellow}}
\hypersetup{
 pdfauthor={Michael Eliachevitch},
 pdftitle={Weekly report},
 pdfkeywords={},
 pdfsubject={},
 pdfcreator={Emacs 28.0.50 (Org mode 9.3.4)}, 
 pdflang={English}}
\begin{document}

\maketitle
\begin{frame}[label={sec:org8fcde09}]{Overview of things I did last week}
\begin{enumerate}
\item fix B\textsubscript{sig} in last weeks reconstruction of \(B \rightarrow D (\rightarrow K\pi\pi) \ell \nu_{\ell}\)
\item reconstruct that on grid with 100 fb\textsuperscript{-1} of MC13
\item in the process figured out how to send basf2 paths from "steering files" with
dependencies (e.g. python packages) to grid via gbasf2
\item started implementing tasks that in b2luigi
\item found out how to send grid jobs from NAF, BAF and from my notebook
\end{enumerate}
\end{frame}


\begin{frame}[label={sec:org956ec8d},fragile]{Update on last weeks MC13 reconstruction}
 \begin{itemize}
\item Felix' b2bii \texttt{rdstar} online reconstruction based on framework functions \& classes
\item instead of trying adopt too Belle II simultaneously, start with more classical
reconstruction script
\item import some helper functions and cuts from \texttt{analysistools} and \texttt{rdstar} piece
by piece to get to know them
\lstset{language=Python,label= ,caption={Code example},captionpos=b,numbers=none,basicstyle=\tiny\ttfamily, xleftmargin=-5pt}
\begin{lstlisting}
pi_charged_cuts = [cut for cut in fsp_utils.charged_pion.cuts if "IDBelle" not in cut]
pi_charged_cuts += ["pionID > 0.1"]
mA.cutAndCopyList("pi+:fsp", "pi+:good", cut_list_to_string(pi_charged_cuts), path=path)
\end{lstlisting}
\end{itemize}
\end{frame}
\begin{frame}[label={sec:orgaa9689f}]{\(m_{D^+}\) plot on 100 fb\textsuperscript{-1}}
\begin{center}
\includegraphics[width=.4\textwidth]{./figures/M_D.pdf}
\end{center}
\end{frame}
\begin{frame}[label={sec:orgafba358}]{B\textsubscript{sig} plots}
\begin{center}
\includegraphics[width=.25\textwidth]{./figures/B_Mbc.pdf}
\end{center}
\begin{center}
\includegraphics[width=.25\textwidth]{./figures/B_deltaE.pdf}
\end{center}
\end{frame}
\begin{frame}[label={sec:orgf335e33},fragile]{How to I send grid jobs without needing the python packages installed there}
 \begin{itemize}
\item import particle classes etc. from \texttt{AnalysisTools} and \texttt{RDStar} \\
\(\rightarrow\) not installed on grid workers
\item produces a basf2.Path object, which itself has no depencencies except basf2
\item idea: serialize the path as a pickle and send it to grid with a wrapper
steering file that just executes the pickle
\item use existing functions \texttt{basf2.pickle\_path}: \texttt{write\_path\_to\_file} and \texttt{get\_path\_from\_file}
\item also possible on CLI with \texttt{basf2 -{}-dump\_path} and \texttt{basf2 -{}-execute\_path}
\end{itemize}
\end{frame}
\begin{frame}[label={sec:org5885a38},fragile]{Next steps}
 \begin{block}{Analysis}
\begin{itemize}
\item online
\begin{itemize}
\item I could keep adding to my own trivial script, e.g. add ROE reco, more
channels, more variables
\item Or: Start adapting Felix' offline reconstruction properly
\end{itemize}
\item offline
\begin{itemize}
\item use framework tools in my notebooks: hists via \texttt{TemplateFitter}, fitting etc.
\item to prepare for high amount of data, adapt framework Dask
\end{itemize}
\end{itemize}
\end{block}
\begin{block}{Grid}
\begin{itemize}
\item Write a b2luigi wrapper to create path pickles, wrapper scripts and send them
to script automatically
\item Pretty straightforward, let \texttt{gbasf2} do the work
\end{itemize}
\end{block}
\end{frame}
\end{document}