% Created 2020-04-28 Di 14:02
% Intended LaTeX compiler: xelatex
\documentclass[aspectratio=169, 12pt]{beamer}
\usepackage{graphicx}
\usepackage{grffile}
\usepackage{longtable}
\usepackage{wrapfig}
\usepackage{rotating}
\usepackage[normalem]{ulem}
\usepackage{amsmath}
\usepackage{textcomp}
\usepackage{amssymb}
\usepackage{capt-of}
\usepackage{hyperref}
\usepackage{xcolor}
\usepackage{physics}
\usepackage{siunitx}
\usepackage{booktabs}
\usepackage[babel]{microtype}
\author{Michael Eliachevitch (\href{mailto:meliache@uni-bonn.de}{meliache@uni-bonn.de})}
\date{\today}
\title{WG1 Meeting: Status and plans}
\subtitle{Adapting Felix Metzners \texttt{b2bii}-based R(D\textsuperscript{(*)}) framework for to Belle II data}
\usepackage{templates/metropolisbonn}
\usepackage{hepnames, hepparticles}
\usepackage[mode=build]{standalone}
\institute{Physikalisches Institut --- Rheinische Friedrich-Wilhelms-Universität Bonn}
\hypersetup{colorlinks, urlcolor=bonnyellow}
\lstset{keywordstyle=\bfseries\color{bonnblue}, commentstyle=\itshape\color{bonnunigrau}, identifierstyle=\color{bonntextgrau}, stringstyle=\color{bonnyellow}}
\hypersetup{
 pdfauthor={Michael Eliachevitch (\href{mailto:meliache@uni-bonn.de}{meliache@uni-bonn.de})},
 pdftitle={WG1 Meeting: Status and plans},
 pdfkeywords={},
 pdfsubject={},
 pdfcreator={Emacs 28.0.50 (Org mode 9.3.6)}, 
 pdflang={English}}
\begin{document}

\maketitle


\begin{frame}[label={sec:orgc132c2e}]{Belle RDStar Analysis Framework by Felix Metzner}
\begin{itemize}
\item \alert{\alert{physics approach:}}
\begin{itemize}
\item \(\rdstar\) analysis with hadronic tag and leptonic signal tau decays
\item use BaBar approach of common M\textsubscript{miss}\textsuperscript{2} and p\textsubscript{\(\ell\)} fit
\end{itemize}
\item \alert{\alert{software software:}}
\begin{itemize}
\item separation into analysis-agnostic and specific pip-installable python packages
\begin{itemize}
\item \href{https://gitlab.etp.kit.edu/fmetzner/AnalysisTools}{AnalysisTools}
\item \href{https://gitlab.etp.kit.edu/fmetzner/RDStar}{RDStar} (my work on fork \href{https://gitlab.etp.kit.edu/meliachevitch/RDStar}{here})
\end{itemize}
\item use \href{https://github.com/nils-braun/b2luigi}{b2luigi} for managing KEKCC
jobs (\href{https://b2luigi.readthedocs.io/en/stable/}{documentation})
\begin{itemize}
\item luigi: software for task managment / pipelining
\item facilitates varying trying parameter variation while ensuring reproducibility
\end{itemize}
\end{itemize}
\end{itemize}
\begin{block}{Problems for Belle II data}
\begin{itemize}
\item difficult to send jobs with complex dependencies to grid
\item luigi did not support grid jobs
\end{itemize}
\end{block}
\end{frame}
\begin{frame}[label={sec:org01a945a},fragile]{My work so far: Technical solutions for working with grid}
 \begin{itemize}
\item to enable working with the grid, instead of sending the whole python projects,
do only send the basf2 paths that they produce
\item exploit \texttt{pickle\_path} module from basf2 to save/load paths from/to disk
\end{itemize}
\begin{center}
\includestandalone[width=0.65\textwidth]{figures/pickled_paths_to_gbasf2/pickled_paths_to_gbasf2}
\end{center}
\end{frame}
\begin{frame}[label={sec:org28a79f3},fragile]{Build a \texttt{gbasf2} backend for \texttt{b2luigi} around that}
 \begin{itemize}
\item uses the \texttt{pickle\_path} functions to send arbitrary basf2 tasks to the grid via luigi
\item code on my own \href{https://github.com/meliache/b2luigi/tree/feature/gbasf2-wrapper-batch-process}{github fork} (\href{https://meliache.github.io/b2luigi/docs/\_build/html/usage/batch.html\#gbasf2-wrapper-for-lcg}{documentation with example}), \href{https://github.com/nils-braun/b2luigi/pull/32}{pull request} pending
\end{itemize}
\begin{block}{Features}
\begin{itemize}
\item send gbasf2 jobs from the basf2 environment (no need to setup gbasf2
manually, but requires that gbasf2 is installed)
\item automatic gbasf2 project submission and download
\item if enabled, automatically reschedule failed jobs until a maximum number of
retries is reached
\end{itemize}
\end{block}
\end{frame}


\begin{frame}[label={sec:org9c6238f}]{Summer plans: Physics}
\begin{itemize}
\item so far I only produced ntuples of limited channels to learn the modular analysis
framework and how gbasf2 works
\item now: combine my tools with Felix framwork to produce proof-of-concept NTuples
with all channels and variables on MC and Data
\item study cut flows
\end{itemize}
\end{frame}

\begin{frame}[label={sec:org38f8181}]{Backup}
\appendix
\end{frame}
\begin{frame}[label={sec:orgdac5cf3},fragile]{Backup: Luigi Gbasf2 example \href{https://meliache.github.io/b2luigi/docs/\_build/html/usage/batch.html\#gbasf2-wrapper-for-lcg}{(documented here)}}
 \lstset{language=Python,label= ,caption= ,captionpos=b,numbers=none,basicstyle=\Tiny\ttfamily, xleftmargin=-5pt}
\begin{lstlisting}
from os.path import join
import b2luigi
from b2luigi.basf2_helper.tasks import Basf2PathTask
import example_mdst_analysis

class MyAnalysisTask(Basf2PathTask):
    batch_system = "gbasf2"
    gbasf2_project_name_prefix = b2luigi.Parameter()
    gbasf2_input_dataset = b2luigi.Parameter(hashed=True)

    mbc_lower_cut = b2luigi.IntParameter()

    def create_path(self):
        mbc_range = (self.mbc_lower_cut, 5.3)
        return example_mdst_analysis.create_analysis_path(mbc_range=mbc_range)

    def output(self):
        yield self.add_to_output(self.gbasf2_project_name_prefix)

class MasterTask(b2luigi.WrapperTask):
    def requires(self):
        input_dataset = join("/belle/MC/release-04-00-03/DB00000757/MC13a/prod00009434/s00/e1003/",
                             "4S/r00000/mixed/mdst/sub00/mdst_000255_prod00009434_task10020000255.root")
        mbc_lower_cuts = [5.15, 5.2, 5.25]
        for mbc_lower_cut in mbc_lower_cuts:
            yield MyAnalysisTask(
                mbc_lower_cut=mbc_lower_cut,
                gbasf2_project_name_prefix="luigiFreeFunctions",
                gbasf2_input_dataset=input_dataset,
                max_event=25,
            )

if __name__ == '__main__':
    master_task_instance = MasterTask()
    n_gbasf2_tasks = len(list(master_task_instance.requires()))
    b2luigi.process(master_task_instance, workers=n_gbasf2_tasks)
\end{lstlisting}
\end{frame}
\end{document}