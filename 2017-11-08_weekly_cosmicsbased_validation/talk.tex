\documentclass[18pt]{beamer}
\usepackage[utf8]{inputenc}
\usepackage{templates/mytemplate}
\usepackage{templates/beamerthemekit}
\usepackage{graphicx}
\usepackage{microtype}
\usepackage{listings}
\usepackage{color}
\usepackage{hyperref}
\usepackage{multicol}
\usepackage{siunitx}
\usepackage{physics}
\usepackage{appendixnumberbeamer}

\title{Estimation of the Track Finding Efficiency using Cosmics Data} 
\subtitle{ETP Belle 2 Weekly Meeting}
\author{\underline{Michael Eliachevitch}}
\date{8 November 2018}
\titleimage{tracks_wide}
\institute{ETP -- KIT}


\begin{document}
  \selectlanguage{english}
  
  \begin{frame}
  \titlepage
\end{frame}


  \begin{frame}
    \frametitle{Split MC Tracks on $\Delta t$ between Hits}
    \begin{itemize}
    \item split \texttt{MCRecoTrack} when time between subsequent \texttt{CDCSimHit}s is larger than chosen $\Delta t$
    \item add new parameter \texttt{SplitAfterDeltaT} to \texttt{TrackFinderMCTruthRecoTracksModule}
    \item \href{https://stash.desy.de/projects/B2/repos/software/pull-requests/737}{pull request} has already been merged
    \item choose high enough $\Delta t$ to only split on passing through SVD volume
    \item<4> method not yet used, still a \textcolor{kit-red100}{WIP}
    \end{itemize}
    \begin{center}
      \includegraphics<1>[width=0.45\textwidth]{figures/delta_t/delta_t_log.pdf}
      \includegraphics<1>[width=0.45\textwidth]{figures/delta_t/delta_t_max_log.pdf}
      \includegraphics<2>[width=0.45\textwidth]{figures/delta_t/delta_t_linear.pdf}
      \includegraphics<2>[width=0.45\textwidth]{figures/delta_t/delta_t_max_linear.pdf}
      \includegraphics<3->[width=0.45\textwidth]{figures/delta_t/delta_t_linear_annotated.pdf}
      \includegraphics<3->[width=0.45\textwidth]{figures/delta_t/delta_t_max_linear_annotated.pdf}
    \end{center}
  \end{frame}

  \begin{frame}
    \frametitle{False Finding Fail Candidate (``Background'')}
    \begin{itemize}
    \item many tracks leave acceptance after passing through SVD volume
    \item this event leaves enough hits in the second hough to pass through my naive cuts on hit content $\rightarrow$ need for more sophisticated methods
    \end{itemize}
    \begin{center}
      \includegraphics[width=0.7\textwidth]{figures/b2display_example_bkgevt.png}
    \end{center}
  \end{frame}


  \appendix
  \section{Backupslides}
  \backupbegin
  
  \begin{frame}
    \begin{center}
      \huge Backup
    \end{center}
  \end{frame}

    \begin{frame}
    \begin{center}
      \frametitle{Example Finding Fail Event}
      \includegraphics[width=0.7\textwidth]{figures/b2display_example_1trackevt.png}
    \end{center}
  \end{frame}

  \begin{frame}[fragile]
    \frametitle{Code for Splitting Tracks in MC Track Finder}
    \begin{lstlisting}[language=C++]
std::vector< std::vector<TimeHitIDDetector> > hitsWithTimeAndDetectorInformationVectors;

if (m_splitAfterDeltaT < 0.0) { // no splitting, vector will only contain a single hitInformation vector
  hitsWithTimeAndDetectorInformationVectors.push_back(hitsWithTimeAndDetectorInformation);
} else { // split on delta t

  std::vector<TimeHitIDDetector>::size_type splitFromIdx = 0; // whenever splitting subtrack, start slice from this index
  for (std::vector<TimeHitIDDetector>::size_type i = 1; i != hitsWithTimeAndDetectorInformation.size(); i++) {

    double delta_t = (std::get<0>(hitsWithTimeAndDetectorInformation[i])
                      - std::get<0>(hitsWithTimeAndDetectorInformation[i - 1]));

    if (delta_t > m_splitAfterDeltaT) {
      // push slice of `hitsWithTimeAndDetectorInformation' between splitFromidx  and previous index
      hitsWithTimeAndDetectorInformationVectors
      .emplace_back(hitsWithTimeAndDetectorInformation.begin() + splitFromIdx,
                    hitsWithTimeAndDetectorInformation.begin() + i);
      splitFromIdx = i;
    }
  }
  // add subtrack after last splitting to list of tracks
  hitsWithTimeAndDetectorInformationVectors
  .emplace_back(hitsWithTimeAndDetectorInformation.begin() + splitFromIdx,
                hitsWithTimeAndDetectorInformation.end());
}
    \end{lstlisting}
  \end{frame}

  \backupend

\end{document}

%%% Local Variables:
%%% mode: latex
%%% TeX-master: t
%%% End:
